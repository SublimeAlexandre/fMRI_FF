\section{Preliminaries}
This is the documentation for the fMRI Experiment on risk perception for Flavia Filimon in 2014. \\ 
It is dedicated to explain the structure of the experimental program and the parameters of the experiment and gives a step by step manual to operate the experiment. \\
The program consists of several files:
\begin{itemize}
    \item trial\_parameters.py: contains all parameters of the experiment. If something can be set, it is set there.
    \item randomization.py: a script to generate the timing and balancing information for the experiment run that are then stored in the run\_parameters folder.
    \item run\_trial.py: the script to start the experiment. 
    \item trial\_classes.py: contains all the classes and routines for presentation of visuals and sound to the participant, the logging of participant input and the output of timestamped information.
    \item timestamp\_comprehension\_v1.py and timestamp\_comprehension\_v2.py: scripts to disassemble the timestamped output from the experiment for the regressor based analysis of fMRI data.
\end{itemize}

\section{Parameters}
The trial\_parameters.py contains ALL parameters that are in use in the experiment (timing, randomization, difficulties, position and size of visual objects etc.). \\
The trial\_parameters.py files is designed to be self explanatory. It is structured in different sections for parameters concerning different parts of the experiment:
\begin{itemize}
    \item Difficulties and Rewards
    \item Parameters for different tasks (Auditory, visual and arithmetic)
    \item Trial timing
    \item Settings for randomization (number of runs, blocks and trials per block, trial types)
    \item Default settings for participant ID and run number
    \item Input keys, start and escape signals
    \item File locations
    \item Eye Tracker settings
    \item Position and size of visual elements
\end{itemize}
For each parameter there is a short explanation of its function.

\section{Randomization and Balancing}
The randomization is done via the randomization.py script. To use it, all parameters must be set within the trial\_parameters.py file. The one opens up a terminal, navigates to the folder where the script is stored and runs it via ``python randomization.py''.
The script creates several sets of experiment parameters (timing, balancing information) and stores them in 'run\_parameters/'. It thereby randomizes the timings of the delay and baseline according to a geometric distribution with given means while making sure, that the given number of ``long'' delays is present for each condition.\\
It also balances the appearance of easy and hard tasks (first or second), the order of trials in terms of small and large EV gaps and the assignment of the high/low EV to the hard/easy task.

\section{How to Use the Program - Step by Step}

These are step by step instructions for the usage of the experiment software:
\begin{itemize}
    \item[1] Make sure the necessary software is installed on the experimental machine. This includes Psychopy, eyelink, pylink, matplotlib and their dependencies.
    \item[2] Make all settings for the experiment within the trial\_parameters.py file.
    \item[3] Open a termal navigate to the location of the experiment program  and run it with ``python run\_trial.py''.
    \item[4] Strap the participant into the machine.
    \item[5] Check the settings for the participant ID, the difficulty settings and the run number, change them if necessary and hit OK.
    \item[5] Wait for the eye tracker initialization (if it is connected).
    \item[6] If the eye tracker is initialized the program waits for the human start signal.
    \item[7] If the human start signal is given, the program waits for the fMRI start signal.
    \item[8] If the fMRI start signal is received, the program subsequently presents the trials to the participant.
    \item[9] After the experiment finished, run the timestamp comprehension routine on the timestamp output via ``python timestamp\_comprehension\_1.py \#timestamp\_file.txt'' or ``python timestamp\_comprehension\_2.py \#timestamp\_file.txt'' to obtain the output files for the different regressors/groups of regressors
        
\end{itemize}

